\documentclass[11pt]{article}

\begin{document}
NAME: Soumil Thakurata

Roll No: 21111064

Stream: BIOMEDICAL ENGINEERING

Subject: Anatomy and Physiology

Topic: Assignment 1


\pagebreak

\section*{Summary of Chapter 1}

\subsection*{\centering Difference Between Anatomy and Physiology}

This chapter is the introduction to the two essential branches of science that a Biomedical Engineer must be familiar with: anatomy and physiology. The difference between the two are also enunciated, which is as follows: anatomy is the science of body structures and the relationship between them (usually studied by the process of dissection), while physiology is the study of body functions, i.e., how the different parts work in practice.

\subsection*{\centering Levels of Organisation in Body Systems}

The different levels of organisation of our body systems are as follows:

1. Chemical Level: This level includes atoms and molecules (the building blocks of all matter), the elements like hydrogen, nitrogen, etc., as well as compounds and molecules formed by multiple elements combinin together via various chemical bonds.

2. Cellular Level: Molecules combine to form living cells, which are the structural and functional units of a human body. There are innumerable types of cells in a human body.

3. Tissue Level: Tissues are a group of cells including the material surrounding them that perform a particular specific function. Humans have four basic types of tissues: epithelial tissue, connective tissue, muscular tissue, and nervous tissue.

4. Organ Level: Different types of tissues are joined together to form an organ. Organs have recognisable shapes and perform specific functions. Some examples of organs are the heart, liver, kidneys, lungs, etc.

5. Organ-System Level: An organ-system consists of several organs that have a common function. For example, the digestive system consists of several organs like the stomach, liver, gall bladder, small intestine, large intestine etc. All these organs work together for the process of digestion.

6. Organismal Level: All the organ-systems taken together constitute a single human body.

\subsection*{\centering Basic Life Processes}

The basic life processes are as follows:

1. Metabolism: It is the term used to describe the sum of all chemical processes in the human body. It is subdivided into catabolism and anabolism.

2. Responsiveness: It is the body's ability to detect and respond to external stimuli. 

3. Movement: It is the motion of individual parts of the body or the body as a whole, for example, running.

4. Growth: Growth is an increase in length or girth of a body, or a tissue.

5. Reproduction: It is an essential process via which new tissue cells and new human beings are born.

\subsection*{\centering Homeostasis}

Homeostasis is the process of maintaining the equilibria or steady of the human body. It is usually regulated via bodily fluids, subdivided into intracellular fluid, extracellular fluid and interstitial fluid. Homeostasis is maintained via feedback systems (both positive and negative), which consist of the following components: a receptor, a control center and an effector. One example of a positive feedback system would be the release of oxytocin during delivery of a child, which aids in contracting the muscles of the uterus more forcefully. One example of a negative feedback system would be the lowering of heart rate when specialised nerve cells called baroreceptors detect a higher blood pressure than normal.

\subsection*{\centering Anatomical Positions, Body Cavities and Imaging Techniques}

The following is a list of anatomical positions with their meanings: supine (body lying face-up), prone (body lying face-down), superior (towards the head), inferior (away from the head), anterior (nearer to the front of body), posterior (nearer to the back of body), medial (nearer to the midline), lateral (farther from the midline), intermediate (between two structures), contralateral(on the opposite side of the body from another structure), ipsilateral (on the same side of the body as another structure), proximal (nearer to
the origination of a structure), distal (farther from the origination of a structure), superficial (nearer to surface of body) and deep (farther from surface of body). The human body has three major cavities: the cranial cavity, the vertebral cavity, the thoracic cavity, and the abdominopelvic cavity.

The following are a list of commonly used medical imaging techniques: radiography, Magnetic Resonance Imaging (MRI), Computed Tomography (CT) scan, ultrasound scanning, Positron Emission Tomography (PET) and endoscopy.












\end{document}